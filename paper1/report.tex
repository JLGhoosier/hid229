\documentclass[sigconf]{acmart}

\usepackage{hyperref}

\usepackage{endfloat}
\renewcommand{\efloatseparator}{\mbox{}} % no new page between figures

\usepackage{booktabs} % For formal tables

\settopmatter{printacmref=false} % Removes citation information below abstract
\renewcommand\footnotetextcopyrightpermission[1]{} % removes footnote with conference information in first column
\pagestyle{plain} % removes running headers

\begin{document}
\title{Big Data in Machine Learning}


\author{ZhiCheng Zhu}
\affiliation{%
  \institution{Indiana University Bloomington}
  \streetaddress{936 S Clarizz Blvd}
  \city{Bloomington} 
  \state{Indiana} 
  \postcode{47401}
}
\email{zhuzhic@iu.edu}


\begin{abstract}

    With the development of IT technology, the world will enter the information age. People also called this “the era of third industrial revolution”. Given the continuous development of the process of the third industrial revolution, all aspects of the traditional way of human society are changing. It can be said that every minute on the internet, have large amounts of new information be produced. With the increasing use of the internet and the increase of network bandwidth. People are making data every moment. It makes the information increasing quite quickly at a phenomenal rate. With so much data becoming available, getting data is not a problem for us anymore but find the right resource from the expanding information becoming a problem for most of the researchers. 
    
\end{abstract}

\keywords{i523, hid229, Big data, Machine Learning, Technology}

\maketitle

\section{Introduction}
The word ``Machine Learning'' first raised by Arthur Samuel, an American pioneer in the computer gaming and artificial intelligence areas. In a broad sense, machine learning is a way to give the machine the ability to learn so that it can complete some task which not done directly by programming. It is a way of using Data to produce a model and then using the model to do prediction. Traditionally, if we want the computer to work, we will give it a series of computer instructions, and then the computer will follow this instruction step by step. The consequence always results by the code which you input before. And you can predict the result. But this way does not work in machine learning. Machine learning does not accept the instructions you entered at all, instead, it will accept the data you entered and applications of machine learning methods to these data sets and finally get produce a result. The core of big data is finding the value from the massive data sets, machine learning is a key technique that can effectively use the data value, for big data, the machine learning is indispensable. On the contrary, for machine learning, the more data will be more likely to improve the accuracy of the model. Therefore, the rise of machine learning is also inseparable from the help of big data. Big data and machine learning are mutually reinforcing and dependent.





\section{The Body of The Paper}

this is the body of the paper



\section{Conclusions}

This is the conclusion



\appendix

%Appendix A




% This next section command marks the start of
% Appendix B, and does not continue the present hierarchy



\begin{acks}

  this is the acknow of th para

\end{acks}

\bibliographystyle{ACM-Reference-Format}
\bibliography{report} 

\end{document}
